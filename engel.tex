\documentclass{subfile}

\begin{document}
	\chapter{Engel Form of Cauchy-Schwarz}\label{ch:engel}
	We mentioned earlier that there is a particular case of Cauchy-Schwarz inequality which is very useful for solving problems. It is known as \textit{Engel form of Cauchy-Schwarz}. Some people also call it \textit{Titu's lemma} or \textit{T2's lemma}. This name became popular among the USA students who attended the IMO training camp after a lecture given by Titu Andreescu at Math Olympiad Summer Program (MOSP) at Georgetown University in June, $2001$. Even though it is a direct consequence of Cauchy-Schwarz inequality, it can be proven independently as well. Moreover, Cauchy-Schwarz inequality can be proven using this result as well.
		\begin{theorem}[Engel form of Cauchy-Schwarz]
			Let $a_1,\ldots,a_n,b_1,\ldots,b_n$ be real numbers. Then
				\begin{align}
					\dfrac{a_1^2}{b_1^2}+\ldots+\dfrac{a_n^2}{b_n^2}
						& \geq\left(\dfrac{a_1+\ldots+a_n}{b_1+\ldots+b_n}\right)^2\label{ineq:engel}
				\end{align}
		\end{theorem}
	
		\begin{proof}
			We will use induction to prove \ref{ineq:engel}. The inequality is trivial for $n=1$. For $n=2$,
				\begin{align*}
					\dfrac{a^2}{x^2}+\dfrac{b^2}{y^2}
						& \geq\dfrac{(a+b)^2}{(x+y)^2}\\
					\iff\dfrac{a^2y^2+b^2x^2}{x^2y^2}
						& \geq\dfrac{(a+b)^2}{(x+y)^2}\\
					\iff
				\end{align*}
		\end{proof}
\end{document}