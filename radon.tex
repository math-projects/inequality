\documentclass[inequalities.tex]{subfile}

\begin{document}
	\section{Radon's Inequality}\label{sec:radon}
	\index{Radon's inequality} \textcite{johann_radon_1913} proves the following generalization of Engel form of Cauchy-Schwarz inequality.
		\begin{theorem}[\itshape Radon's Inequality]\label{thm:radon}
			Let $b_{1},\ldots,b_{n}$ be positive real numbers and $a_{1},\ldots,a_{n};m$ be non-negative real numbers. Then
				\begin{align*}
					\dfrac{a_{1}^{m+1}}{b_{1}^{m}}+\ldots+\dfrac{a_{n}^{m+1}}{b_{n}^{m}}
						& \geq \dfrac{(a_{1}+\ldots+a_{n})^{m+1}}{(b_{1}+\ldots+b_{n})^{m}}
				\end{align*}
		\end{theorem}
	\textcite{beckenbach_1950} proves a similar result. For a vector of positive real numbers $\mathbf{a}$ and a real number $r$, let us define
		\begin{align*}
			\mathfrak{R}_{r}(\mathbf{a})
				& = \dfrac{\sum\limits_{i=1}^{n}a_{i}^{r}}{\sum\limits_{i=1}^{n}a_{i}^{r-1}}
		\end{align*}
	Note that this is actually a weighted arithmetic mean of $\mathbf{a}$ with the weight vector
		\begin{align*}
			\omega
				& = \left(\dfrac{a_{1}^{r-1}}{\sum\limits_{i=1}^{n}a_{i}^{r-1}},\ldots,\dfrac{a_{n}^{r-1}}{\sum\limits_{i=1}^{n}a_{i}^{r-1}}\right)
		\end{align*}
	This is also a generalization of arithmetic, geometric and harmonic means. We get the usual arithmetic and harmonic means if we set $r=1$ and $-1$ respectively. $r=\frac{1}{2}$ gives us the geometric mean for $n=2$.
		\begin{theorem}[\itshape Beckenbach's inequality]\label{thm:beckenbach}
			Let $r$ be a real number. Then
				\begin{align*}
					\mathfrak{R}_{r}(\mathbf{a}+\mathbf{b})
						& 
							\begin{cases}
								\leq \mathfrak{R}_{r}(\mathbf{a})+\mathfrak{R}_{r}(\mathbf{b}) &\mbox{if }1\leq r\leq 2\\
								\geq \mathfrak{R}_{r}(\mathbf{a})+\mathfrak{R}_{r}(\mathbf{b}) &\mbox{if }0\leq r\leq 1 
							\end{cases}
				\end{align*}
			Equality occurs if and only if $r=1$ or $\mathbf{a}$ is proportional to $\mathbf{b}$.
		\end{theorem}
	\textcite{yang_2002} proves the following generalization of \nameref{thm:radon}.
		\begin{theorem}[\itshape Generalized Radon's Inequality]\label{thm:genradon}
			Let $b_{1},\ldots,b_{n}$ be positive real numbers and
				\begin{align*}
					a_{1},\ldots,a_{n};r,s
				\end{align*}
			be non-negative real numbers such that $r\geq s+1$. Then
				\begin{align*}
					\dfrac{a_{1}^{r}}{b_{1}^{s}}+\ldots+\dfrac{a_{n}^{r}}{b_{n}^{s}}
						& \geq \dfrac{(a_{1}+\ldots+a_{n})^{r}}{n^{r-s-1}(b_{1}+\ldots+b_{n})^{n}}
				\end{align*}
		\end{theorem}
	\textcite{yongtao_xian_xiao_2018} proves the equivalence between some well known inequalities.
		\begin{theorem}[\itshape Equivalence of inequalities]
			The following inequalities are mutually equivalent.
				\begin{enumerate}[\itshape i.]
					\item Bernoulli's inequality
					\item The weighted arithmetic-geometric mean inequality
					\item H\"{o}lder's inequality
					\item The weighted power mean inequality
					\item Minkowski's inequality
					\item Radon's inequality
				\end{enumerate}
		\end{theorem}
\end{document}