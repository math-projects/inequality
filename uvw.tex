\documentclass{subfile}

\begin{document}
	\section{The UVW/PQR/ABC Method}\label{sec:uvw}
	The UVW method was popularized by \textcite{rozenberg_2011}. It has also been known as the ABC method or the PQR method although it is unclear what exactly the origin of this method is. Although some people believe it was originated at Vietnam where it was called the ABC (Abstract concreteness) method. Currently, \textcite{knudsen} is the most popular version of this technique but at the core, they all use the same idea. We will try to explain this method as clearly as possible with examples.

	Consider an inequality in three variables $a,b,c\in\mathbb{R}$. If the expression is symmetric on $a,b,c$, then the initial idea of PQR method was to write
		\begin{align*}
			a+b+c
				& = p\\
			ab+bc+ca
				& = q\\
			abc
				& = r
		\end{align*}
	so that $a,b,c$ are the roots of the equation
		\begin{align*}
			x^{3}-px^{2}+qx-r
				& = 0
		\end{align*}
	However, nowadays the most popular way to convert these is to write
		\begin{align*}
			a+b+c
				& = 3u\\
			ab+bc+ca
				& = 3v^{2}\\
			abc
				& = w^{3}
		\end{align*}
	Hence, the name UVW. While working with such transformations, be careful not to assume $v^{2}\geq 0$ by default. We have to consider the case where $v^{2}$ is negative as well. We do not know for sure that $a,b,c$ are all positive unless it is stated specifically. However, the following result is very nice when they are indeed positive. We will use the notations as stated above throughout this section.
		\begin{theorem}
			If $a,b,c\geq 0$, then $u\geq v\geq w$.
		\end{theorem}
\end{document}